\documentclass[12pt,a4paper]{report}
\usepackage[left=1.57in,top=1.18in,bottom=1in,right=1in]{geometry}
\usepackage{graphicx}
\usepackage{amsmath}
\usepackage{listings}
\usepackage{caption}
\usepackage{subcaption}
\usepackage{csquotes}
\usepackage{float}
\usepackage{color}
\usepackage{gensymb}
\usepackage{multirow}
\usepackage{array}
\usepackage{mathtools}
\DeclarePairedDelimiter{\abs}{\lvert}{\rvert}
\usepackage[toc,page]{appendix}
\usepackage{setspace}
\singlespacing
\captionsetup[figure]{labelfont={bf},labelformat={default},labelsep=period,name={Fig.}}

%%%%%%%%%%%%%%%%%%%%%%%%%%%%%%%%%%%%%%%%%%%%%%%%%%%%%%%%%
\begin{document}

%%%%%%%%%% Title-Pages %%%%%%%%%%%%%%%
%%%%%%%%%%%%%%%%%%%%%%%%%%%%%%%%%%%%%%
\begin{titlepage}
	\centering
	\includegraphics[width=0.4\textwidth]{Figures/Logo.jpeg}\par\vspace{1cm}
	\vspace{1cm}
		{\scshape\Large\bfseries Suspicious Bangla Text Detection Using Machine Learning  \par}
	\vspace{2cm}
	 
	\textbf{Omar Sharif}\par 
	\vspace{.5cm}
	\textbf{ID: 1304003}\par 
	\vspace{3cm}
	\textbf{October, 2018}
	
	\vspace{4cm}
	{\Large\bfseries Department of Computer Science \& Engineering\par}
	{\large\bfseries Chittagong University of Engineering \& Technology\par}
	{\bfseries Chittagong-4349, Bangladesh.}

	\vfill
\end{titlepage}

\begin{titlepage}
	\centering
	%\includegraphics[width=0.15\textwidth]{example-image-1x1}\par\vspace{1cm}
	{\scshape\LARGE\bfseries Bachelor of Science in Computer Science and Engineering \par}
	\vspace{7cm}
	{\scshape\Large\bfseries Suspicious Bangla Text Detection Using Machine Learning \par}
	\vspace{3cm}
	{\scshape\bfseries Omar Sharif\par}
	{\scshape\bfseries ID: 1304003\par}
	
	\vspace{8cm}
	{\bfseries Department of Computer Science \& Engineering\par}
	{\bfseries Chittagong University of Engineering \& Technology\par}
	{\bfseries Chittagong-4349, Bangladesh.}

	\vfill

% Bottom of the page
	{\large October 29, 2018\par}
\end{titlepage}

\begin{titlepage}
	\centering
	{\scshape\Large\bfseries Suspicious Bangla Text Detection Using Machine Learning Algorithm \par}
	\vspace{1cm}
	\includegraphics[width=0.4\textwidth]{Figures/Logo.jpeg}\par\vspace{1cm}
	\vspace{1cm}
	 This thesis is submitted in partial fulfillment of the requirement for the degree of Bachelor of Science in Computer Science \& Engineering.
	
	\vspace{2cm}
	\textbf{Omar Sharif}\par 
	\vspace{.5cm}
	\textbf{ID: 1304003}\par 
	
	\vspace{2cm}
	Supervised by\par 
	Prof. Dr. Mohammed Moshiul Hoque\par 
	Department of Computer Science \& Engineering (CSE)\par 
	Chittagong University of Engineering \& Technology (CUET)
	
	\vspace{1.5cm}
	{\Large\bfseries Department of Computer Science \& Engineering\par}
	{\large\bfseries Chittagong University of Engineering \& Technology\par}
	{\bfseries Chittagong-4349, Bangladesh.}

	\vfill
% Bottom of the page
	{\large October 29, 2018\par}
\end{titlepage}

\pagenumbering{roman}
\setcounter{secnumdepth}{-1}

\section*{}
The project titled \enquote{{\bfseries Suspicious Bangla Text Detection Using Machine Learning}}, submitted by ID No. 1304003, Session 2016-2017 has been accepted as satisfactory in fulfillment of the requirement for the degree of Bachelor of Science in Computer Science \& Engineering (CSE) as B.Sc. Engineering to be awarded by the Chittagong University of Engineering \& Technology (CUET).\par 
\vspace{1.5cm}
{\hfil \Large\bfseries Board of Examiners}\par
\vspace{1.5cm}
 $1.$\underline{\hspace{8cm}}\hspace{3.5cm}Chairman\par
 \vspace{.5cm}
 Professor Dr. Mohammed Moshiul Hoque \hspace{4cm}(Supervisor)\par 
 Department of Computer Science \& Engineering (CSE)\par
 Chittagong University of Engineering \& Technology (CUET)\par 
\par 
\vspace{1.5cm}
 $2.$\underline{\hspace{8cm}}\hspace{3.5cm}Member\par
 \vspace{.5cm}
 Professor Dr. Mohammad Shamsul Arefin \hspace{4cm}(Ex-officio)\par
 Head of the Department\par
 Department of Computer Science \& Engineering (CSE)\par 
 Chittagong University of Engineering \& Technology (CUET)\par 
\par 
\vspace{1.5cm}
 $3.$\underline{\hspace{8cm}}\hspace{3.5cm}Member\par
 \vspace{.5cm}
 Professor Dr. Mohammad Shamsul Arefin \hspace{4cm}(External)\par 
 Head of the Department\par 
 Department of Computer Science \& Engineering (CSE)\par
 Chittagong University of Engineering \& Technology (CUET)\par 
\clearpage


\section*{}
{\scshape\Large\bfseries\hfil Statement of Originality \par}
\vspace{1cm}
\noindent
It is hereby declared that the contents of this project is original and any part of it has not been submitted elsewhere for the award if any degree or diploma.\par
\vspace{7cm}
\noindent
\underline{\hspace{5.5cm}} \hspace{3cm} \underline{\hspace{5.5cm}}
\par 
\vspace{.5cm}
\noindent
{\bfseries Signature of the Supervisor\hspace{3cm}Signature of the Candidate}
\par 
\vspace{.5cm}
\noindent
{\bfseries Date: \hspace{7.4cm}Date:}
\clearpage

%%%%%%%%%% Acknowledgement %%%%%%%%%%%
%%%%%%%%%%%%%%%%%%%%%%%%%%%%%%%%%%%%%%
\section{\hfil Acknowledgment \hfil}
\vspace{1cm}
The satisfaction that accompanies the successful completion of this work would be incomplete without the mention of people whose ceaseless cooperation made it possible, whose constant guidance and encouragement crown all efforts with success. I am grateful to my honorable project Supervisor Dr. Mohammed Moshiul Hoque, Professor, Department of Computer Science and Engineering, Chittagong University of Engineering and Technology, for the guidance, inspiration and constructive suggestions which were helpful in the preparation of this project. I also convey special thanks and gratitude to all my respected teachers of the department. I would also like to thank friends and the staffs of the department for their valuable suggestion and assistance that has helped in successful completion of the project.
\clearpage

%%%%%%%%%% Abstract %%%%%%%%%%%%%%%%%%
%%%%%%%%%%%%%%%%%%%%%%%%%%%%%%%%%%%%%%
\section{\hfil Abstract \hfil}
\vspace{1cm}
%%Writing will start from here

\setcounter{secnumdepth}{3} % Reenable the numbering of sections etc. 
\clearpage % Now the table of contents

%%%%%%%%% List of Different Contents %%
%%%%%%%%%%%%%%%%%%%%%%%%%%%%%%%%%%%%%%%
\tableofcontents
\listoffigures
\listoftables
\clearpage
\pagenumbering{arabic}

\onehalfspacing  %%%% One Half spacing  start from here %%%%%
%%%%%% Chapter 1 Introduction %%%%%%%%%%%%%%
%%%%%%%%%%%%%%%%%%%%%%%%%%%%%%%%%%%%%%%%%%%%
\chapter{Introduction}

\thispagestyle{empty}
The original purpose of our project was to develop a system for detecting suspicious Bangla texts using supervised machine learning. Digitization has changed the way we process and analyze information. There is an exponential increase in online availability of information. From web pages to emails, science journals, e-books, learning content, news and social media are all full of textual data. Text classification performs an essential role in various applications that deals with organizing, classifying, searching and concisely representing a significant amount of information.\par
\vspace{.2cm}
Our system classifies a piece of information into suspicious and non-suspicious category. This system can be used by counter terrorism agencies for analysis of information.

\section{Text Classification}
Text classification is the task of assigning a text into a set of predefined classes automatically. Because of the rapid growth of online information, text classification has become more challenging and more important as well. Text classification can be described into two categories.
\subsection{Supervised Classification}
In supervised classification of text classification categories are defined. It works on training and testing principle. During training phase, the machine learning algorithm works on the label data. The algorithm is trained on labeled data set and gives the desired output. During testing phase, unobserved data are fed into algorithm and classifier classifies them based on the knowledge of training phase.
\subsection{Unsupervised Classification}
In unsupervised classification of text classification categories are not defined. Here machine learning algorithm try to discover natural structure in data. The algorithm looks for similar patterns and structures in the data points and groups them into clusters. The classification of the data is done based on the clusters formed.\par
One can also apply some other ways to classify text such as Custom Text Classification, Semi-Supervised Text Classification etc.

\section{Motivation}
With the amount of text files on internet increases exponentially each day, the volume of information available online continues to expand. A great number of researchers focused in the area of counter terrorism after the disastrous events of  9/11 trying to predict terrorist plans from suspicious communication.Because most of communications occur based on text, if we able to predict either a text is suspicious or not suspicious it will be very helpful for the law enforcement agencies to find the perpetrator and stop terrorist event.\par 
\vspace{.3cm}
As far we know, no system is developed for detecting suspicious Bangla text. It is very important for the safety of Bangladeshi People to develop a system   which can detect suspicious communications in Bangla. It is quite impossible for a human being to analyze millions of texts. So our goal is to develop a learned classifier which is able to classify a text as ‘Suspicious’ or ‘Not Suspicious’. This motivated us to work on this area.

\section{Challenges}
We have to face several challenges to implement this system. Some of this challenges are listed below, 

\begin{itemize}
    \item First challenge, for the implementation of this system the most challenging task was to develop a dataset which can be used by our learning algorithm. We know that for any machine learning algorithm a well-furnished dataset is a key. Bengali is a low resource language and there are no good quality corpora is available for research purpose. We try to overcome this problem by collecting a large amount of text from different Bengali source and social media. We collect about 4000 suspicious text and it takes us around six months to prepare this dataset.
    
    \item The second challenge was to label the collected data into suspicious and non-suspicious category. We use supervised machine learning algorithm so it was very important to label the texts correctly. If we use wrong data to train the classifier, then performance will decrease. Semantic feature is used to classify text properly. Semantic features represent the basic conceptual components of meaning for any lexical item. An individual semantic feature constitutes one component of a word’s intension, which is the inherent sense or concept evoked.
    
    \item Hardware support was another vital challenge for our system. For getting better accuracy we have to process huge amount of texts. We set up a high computation power hardware which able to process large amount of texts.
\end{itemize}
%\clearpage

\section{Applications of Text Classification}
Classification is one of the essential parts of Text Analysis. Text classification or Text Categorization is the activity of labeling natural language texts with relevant categories from a predefined set. A lot of current and emerging applications of text classification are available in this era of commercial digitization. Some of the applications are,
\begin{itemize}
    \item Platforms such as E-commerce, news agencies, blogs, content curators, directories can use automated technologies to classify and tag content and products. Tagging content or products using categories as a way to improve browsing or to identify related contents on a website.
    
    %\item Text classification can also be used to automate CRM (Customer Relationship Management) tasks. The text classifier is highly customizable and can be trained accordingly. The CRM tasks can directly be assigned and analyzed based on importance and relevance. It reduces manual work and thus is high time efficient.
    
    \item Text Classification of content on the website using tags helps to crawl website easily which ultimately helps in SEO. Additionally, automating the content tags on website and app can make user experience better and helps to standardize them. Text classification can be used to automate and speed up this process.
    
    \item Text classification can be used for analysis of sentiment. It is task of determining sentiment of a user based on the content of the text.
    
    \item Academia, law practitioners, social researchers, government, and non-profit organization can also make use of text classification technology. As these organizations deal with a lot of unstructured text, handling the data would be much easier if it is standardized by categories/tags.
    
   % \item Email spam filtering is one of the most common application of text classification which classifies an email into spam and not spam category.
\end{itemize}

\section{Contributions of the Work}
The primary focus of our project is to develop a system which can detect suspicious Bangla text using machine learning. The contributions are as follows,
\begin{enumerate}
    \item A system is designed that can detect suspicious Bangla text based on machine learning algorithm.
    
    \item A corpus is developed which contains approximately 4000 suspicious Bangla text collected from different online and offline resources.
    
    \item The proposed system is implemented to detect suspicious text.
\end{enumerate}
%\clearpage

\section{Organization of the Thesis}
This report is organized into six chapters. Chapter one contains some introductory readings
on text classification, some challenges of implementation of our work, motivation of our work, motivation of our work and the contributions we made. Chapter two contains brief discussion on previous works that is already implemented, their limitations and their role on text classification using machine learning. Chapter three describes proposed system with necessary diagrams. An overall system architecture is given on this chapter. In chapter four our implementation of the project in details have been illustrated. Chapter five focuses on the experimental results of the system. Evaluation measures and results of our system is described in this chapter. Chapter six consists of conclusion with the summary of our system and
the future plan of our system
\clearpage

%%%%%%%%%% Chapter 2 Literature Review %%%%%%%%%%%%
%%%%%%%%%%%%%%%%%%%%%%%%%%%%%%%%%%%%%%%%%%%%%%%%%%%
\input{LiteratureReview.tex}
    
%%%%%%%%% Chapter 3 Proposed Methodology %%%%%%%%%%%%
%%%%%%%%%%%%%%%%%%%%%%%%%%%%%%%%%%%%%%%%%%%%%%%%%%%%%
\chapter{Proposed Methodology}
\thispagestyle{empty}
In this chapter we will discuss about our proposed methodology and learn about every module of our system. We will find short mathematical insight of each learning algorithm used in our system. Finally overall system is summarized with an example.
\section{Proposed Text Classification System}
The key objective of our project is to design a system that can classify suspicious and non-suspicious text. \textbf{Fig} \ref{fig:proposed_model} shows an abstract view of our system.
\vspace{0.5cm}
\begin{figure}[h!]
\centering
  \includegraphics[scale=0.6]{Figures/proposed_model.PNG}
  \caption{Proposed Model of Suspicious Text Detection}
  \label{fig:proposed_model}
\end{figure}
 
 \section{Training Text Corpus}
 Supervised learning involves using a set of training examples that make up the training data. In linguistics, a corpus or text corpus is a large and structured set of texts nowadays usually electronically stored and processed. For a machine learning algorithm a well corpus is essential to perform according to expectation. In the corpus each training example consists of an input text and desired output value. An algorithm is then applied to the data to produce a classifier which will determine the correct output for any further valid input values. In Bengali language processing there is no corpus available which contains suspicious text. But as ours is a supervised classification system, our suspicious text detector needs a training corpus which consist of text and labels indicating whether a text is suspicious or non-suspicious. To implement this system we collect a large amount of Bangla  suspicious text. This texts are collected from online blogs, newspaper, Facebook post and other type of electronic source. Training corpus is divided into positive set (Suspicious text) and negative set (non-suspicious text). The accuracy of the learning algorithms depend on uniqueness of training examples.
 
 \section{Preprocessing}
 In natural language processing, data preprocessing is an essential step. Data preprocessing is a data mining technique that involves transforming raw data into an understandable format. Real world data is often incomplete, inconsistent, and/or lacking in certain behaviors or trends, and is likely to contain many errors. Data preprocessing is a proven method of resolving such issues. Data preprocessing is required to fill in missing values, smooth noisy data, identify or remove the outliers, and resolve inconsistencies.
 Words with no significance must be removed from the text. In our system all texts are preprocessed by the preprocessor. We use main bodies of the text to train suspicious text detector. We are going to represent our text document by list of words and their frequencies. We have a stop word list which consist of words that make no contribution to classify text. In our preprocessing section, such word will be removed by matching with stop word list. It will be very helpful to increase efficiency of the system because redundant will slow our system and increase computational complexity of the system.
 
\section{Feature Extraction}
Word frequencies will be used as feature in our proposed system. Word frequencies are used as features quite often, and although they are usually considered a basic feature, they can prove to be effective.
Terms associated with feature extraction of our system are described shortly.
\subsection{CountVectorizer}

CountVectorizers used to learn the vocabulary of a set of texts and then transform them into a data-frame that can be used for building models. CountVectorizer takes few parameter that is important for extracting feature accurately.

\subparagraph{Stop Words :}
CountVectorizer just counts the occurrences of each word in its vocabulary, extremely common words stop words will become very important features while they add little meaning to the text. In our system we do not take those words into account which improves system accuracy.
\subparagraph{Min-DF, Max-DF :}
These parameters are the minimum and maximum document frequencies words/n-grams must have to be used as features. If either of these parameters are set to integers, they will be used as bounds on the number of documents each feature must be in to be considered as a feature. If either is set to a float, that number will be interpreted as a frequency rather than a numerical limit.
\subparagraph{Max-Features :}
\texttt{max\_features} parameter is used to limit maximum number of feature used by the model. In our system we take most frequent 1000 words by using this parameter to reduce time and storage complexity.

\subsection{Word2vec}
Word2vec is used to train different learning model. Word2vec takes as its input a large corpus of text and produces a vector space, typically of several hundred dimensions, with each unique word in the corpus being assigned a corresponding vector in the space. Word vectors are positioned in the vector space such that words that share common contexts in the corpus are located in close proximity to one another in the space. Accuracy increases overall as the number of words used increases, and as the number of dimensions increases. But we should be careful about associated complexities. 

\subsection{TF-IDF}
The tf–idf is the product of two statistics, term frequency and inverse document frequency. There are various ways for determining the exact values of both statistics.

\subparagraph{Term Frequency}$tf(t,d)$, the simplest choice is to use the raw count of a term in a document, i.e., the number of times that term $t$ and occurs in document $d$.If we denote the raw count by $f_{t,d}$, then the simplest $tf$ scheme is $tf(t,d) = f_{t,d}$. Term frequencies can be calculated in  different ways,
\begin{enumerate}
    \item Term Frequency : 
    \begin{equation}
        \dfrac{f_{t,d}}{\sum_{t'\epsilon d}^{}f_{t',d}}
    \end{equation}
    \item Logarithmically scaled frequency :
    \begin{equation}
         tf(t,d)=\log(1+f_{t,d})
    \end{equation}
   
    \item To prevent bias for longer documents following equation is used:
    \begin{equation}
        tf(t,d) = 0.5 + 0.5 * \frac{f_{t,d}}{\max{(f_{t',d}\colon t'\epsilon d)}}
    \end{equation}
\end{enumerate}

\subparagraph{Inverse Document Frequency}
is a measure whether a term is common or rare across all document. IDF can be calculated by following equation,
\begin{equation}
    idf(t,d) = \log \frac{N}{\abs{1+(d\epsilon D\colon t\epsilon d)}}
\end{equation}
\noindent
%\vspace{0.5cm}
$N$ : Total number of documents in the corpus $N = \abs{D}$\newline
$(d \epsilon D\colon t\epsilon d)$ : \textrm{Number of document where term $t$ appears.}\newline
\vspace{0.5cm}
Now $tf-idf$ is calculated as,
\begin{equation}
    tfidf(t, d, D) = tf(t, d)*idf(t, D)
\end{equation}
So, Final weighting scheme of $tf-idf$ is,
\begin{equation}
        tfidf(t, d, D)=(0.5 + 0.5 * \frac{f_{t,d}}{\max{(f_{t',d}\colon t'\epsilon d)}})*\log \frac{N}{n_t}
\end{equation}
\clearpage

\section{Classification Algorithm}
After extracting feature from the text, now these features are used to train our machine learning model. Different classification algorithms have been used for training purpose.

\subsection{Intuition of Naive Bayes Classifier}
Naive Bayes is a simple technique for constructing classifiers using feature values. All naive Bayes classifiers assume that the value of a particular feature is independent of the value of any other feature, given the class variable. \textbf{Fig} \ref{fig:NBC} shows decision boundary for Naive Bayes classifier.

\begin{figure}[h!]
    \centering
    \includegraphics[scale=0.4]{Figures/naive_bayes.png}
    \caption{Naive Bayes Classifier}
    \label{fig:NBC}
\end{figure}

%%% Citation ache %%%%%%
As a popular classification algorithm, Naive Bayes algorithm [4] will be used in our system. It can be defined as Bayes theorem with a conditional independency assumption that all variables $A_{1},A_{2},...,A_{n}$ in a given category $C$ are conditionally independent with each other given $C$. 
According to Bayes rule for a text document $(T)$ and class $(C)$ we can write,$$ P(C|T) = \frac{P(T|C)P(C)}{P(T)}$$
The class we are looking for to assign this document is out of all classes the one that maximizes the probability of that class given the document.
\begin{equation}\label{eq:11}
 \begin{aligned}
     C_{MAP} & = argmax P(C|T) \\     & = argmax \frac{P(T|C)P(C)}{P(T)}\\
    & = argmax {P(T|C)P(C)}
\end{aligned}
\end{equation}
So final equation for Naive Bayes Classifier is,
\begin{equation}
     C_{MAP} = argmax P(X_{1},X_{2},...,X_{n}|C)P(C)
\end{equation}

\subsection{Intuition of Support Vector Machine}
Support vector machines are supervised learning models with associated learning algorithms that analyze data used for classification and regression analysis.  A support vector machine constructs a hyperplane or set of hyperplanes in a high- or infinite-dimensional space, which can be used for classification, regression, or other tasks like outliers detection. A good separation is achieved by the hyperplane that has the largest distance to the nearest training-data point of any class. \textbf{Fig} \ref{fig:SVM} shows decision boundary for linear SVM.


\begin{figure}[h!]
    \centering
    \includegraphics[scale=0.4]{Figures/svm.png}
    \caption{Support Vector Machine}
    \label{fig:SVM}
\end{figure}

SVM can perform linear classification as well as non-linear classification using kernel trick. For linear classification cost function can be written as,
\begin{equation}
    \label{cost_function_svm}
    \frac{1}{n}\sum_{i=1}^{n}\max{(0,1-y_{i}(w.x_{i}-b))} + \lambda (\lVert \mathbf{w} \rVert)^2
\end{equation}

The parameter $\lambda $ in equation 3.11 determines the trade-off between increasing the margin-size and ensuring that the samples lie on the correct side of the margin.Thus, for sufficiently small values of $\lambda $, the second term in the loss function will become negligible, hence, it will behave similar to the hard-margin SVM, if the input data are linearly classifiable, but will still learn if a classification rule is viable or not.
\par\noindent
For non-linear classification with SVM kernel trick is used. Some common kernel tricks used in machine learning are,
\begin{itemize}
    \item Gaussian radial basis function:
    \begin{equation}
         k(x_{i}, x_{j}) = \exp{(-\gamma(\lVert \mathbf{x_{i}-x_{j}} \rVert)^2)}
    \end{equation}
    \item Polynomial Kernel (homogeneous):
    \begin{equation}
        k(x_{i}, x_{j}) = (x_{i}\cdot x_{j})^d
    \end{equation}
     \item Polynomial Kernel (inhomogeneous):
     \begin{equation}
         k(x_{i}, x_{j}) = (x_{i}\cdot x_{j}+1)^d
     \end{equation}
    \item Hyperbolic tangent:
    \begin{equation}
       k(x_{i}, x_{j}) = tanh(kx_{i}\cdot x_{j}+c) 
    \end{equation}
\end{itemize}
Above equations expresses the kernel trick which are really important for classification and regression analysis.

\subsection{Intuition of Logistic Regression}

%%%%%%%%%%% Chapter 4 Implementation %%%%%%%%%%%%%%%
%%%%%%%%%%%%%%%%%%%%%%%%%%%%%%%%%%%%%%%%%%%%%%%%%%%%
\input{Implementation.tex}

%%%%%%%%%% Chapter 5 Experimentation %%%%%%%%%%%%%%
\chapter{Experimentation}
\thispagestyle{empty}
In this chapter we will discuss about our dataset preparation process. We will learn about different evaluation measures. Finally by using this evaluation matrices we will evaluate the result of our system.

\section{Dataset}
Behind the success of any machine learning algorithm dataset is the key. Unfortunately, when we start this project there were no dataset available which can be used in our system. It was a very challenging task for us to build a corpus which contains a large amount of suspicious text. It took us approximately  six month to a build a well furnished  suspicious text corpus containing 4000 suspicious text. We collect non suspicious data from a pre-build corpus[16], thanks to them. At the time of collecting data we were very careful because as a machine leaning model our system learn form what we give to it. If we give wrong data then final output we will not be accurate. \par \vspace{0.5cm}\noindent 
We mark a text as suspicious if it satisfies any one of the following criteria.

\begin{itemize}
    \item Texts contain words which hurts our religious              feelings.
    \item Texts which instigate people against government.
    \item Texts which instigate people against law enforcement        agencies.
    \item Texts which instigate a community without any reason.
    \item Texts which instigate our political parties. 
    
\end{itemize}
As dataset is collected manually it may have some inconsistency.

\end{document}

 







